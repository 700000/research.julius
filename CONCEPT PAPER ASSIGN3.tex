\documentclass[10pt]{article}
\begin{document}
\author{}
 MAKERERE UNIVERSITY COLLEGE OF COMPUTING AND INFORMATION SCIENCE
 \\\
 THE CONCEPT PAPER ON DATA COLLECTION.
 \title{} 
 STUDENTS RESULTS COMPLAINT
\\\ 
 OCHAN JULIUS JOSHUA
 \section{Introduction}
 \paragraph{In consideration to the call of the government for a large scale transformation to open governance in all of its entities, the digitized approach for records management is determined as the key factor to materialize the mandate. In such regard, I also find the mandate very essential and agree that it does not only address the said call but, in one way or another helps the university to level-up its performances in providing public services. Therefore, I decided to take the change from old and inefficient to a new and digitized framework of results complaint management to be relevant and inevitable.}
 
\section{Background.}
\paragraph{Barrett (1999) notes that in an effort to efficiently document and maintain accountability data, schools are relying more on technology in the form of Student Management Information Systems (SMIS). The usual manual process has now reached a level where it is difficult for the available man power to cope with the magnitude of examination work, in the given time. The imbalance between man power availability and the magnitude of work to be done in processing means students have to wait for a long time for a response if any and most students report not getting responses which makes them redo the course unit because of delayed response or no response at all. Hence, the need to evolve a computerized process that will effectively and efficiently capture all the important data associated with result complaint lodging and processing within the University more so CIS.my research focuses on the design and implementation of an application where student results complaints can be lodged and responses sent to students on time. Students can access this software from anywhere as long as the computer they are working on is on the network same as the application server.}
 
   \section{Problem Statement}
   \paragraph{Makerere University is now moving towards the highest level of its automation. This means large number of students hence increased work for staff. Because of this, the college and the student at large have been facing a problem of delayed response to the students’ results complaints which has left many affected students to redo their respective course units because sometimes their complaints are left unattended to by staff due to a lot of work pending on their desk. Therefore RCMS would be more efficient to solve the above issue.}
\section{OBJECTIVES}
\subsection{Main Objective}
\paragraph{To develop an online Results Complaints Management System (RCMS) that would update both the student and college staff on the complaint lodged.}
  \subsection{Specific Objectives.}
 \begin{description}
 \item[]	To analyze data collected and come up with specific requirements of the system.
  \item[]
  To analyze the way students’ results complaints was handled by the academic registrar, head of department and lecturers, by determining the information needs of each group through interviews with the end users of the system.
  \item[]
  	To design system that integrates the student’s results complaint process.
   \item[]
   	To test and validate computerized/short message based system that meets the design specifications.
 \end{description}
\section{Scope}
\paragraph{This study covered Makerere University School of Computing and Information Technology (SCIT) the student’s body. In order to their concern on the existing system and see way forward to come up with something convenient and easy to use and effective.}
\section{Methodology }
\paragraph{This section comprises of research/project design which describes the tools, instruments, approaches, processes and techniques, major algorithms and data structures to be employed in the research study, data collection, analysis, synthesis, design, logical flow, implementation, testing, and validation. All the above mentioned is what I will follow to come up with a sound research.}
\section{Significance}
\paragraph{Provision of improved result complaint services to the user through fast, timely and convenient result submission.}
\paragraph{Reduction of the costs incurred by the college time to time in paying the very many clerks employed for the sake of the success of the manual system.}

\end{document}